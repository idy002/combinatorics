\documentclass[12pt]{article}
\usepackage{amsmath}
\usepackage{amssymb}
\usepackage{amsthm}
\providecommand{\abs}[1]{\lvert#1\rvert}
\providecommand{\norm}[1]{\lVert#1\rVert}

\newtheorem{thm}{Theorem}
\newtheorem{lemma}[thm]{Lemma}
\newtheorem{fact}[thm]{Fact}
\newtheorem{cor}[thm]{Corollary}
\newtheorem{eg}{Example}
\newtheorem{ex}{Exercise}
\newtheorem{defi}{Definition}
\theoremstyle{definition}
\newtheorem{hw}{Problem}
\newenvironment{sol}
  {\par\vspace{3mm}\noindent{\it Solution}.}
  {\qed}

\newcommand{\ov}{\overline}
\newcommand{\cb}{{\cal B}}
\newcommand{\cc}{{\cal C}}
\newcommand{\cd}{{\cal D}}
\newcommand{\ce}{{\cal E}}
\newcommand{\cf}{{\cal F}}
\newcommand{\ch}{{\cal H}}
\newcommand{\cl}{{\cal L}}
\newcommand{\cm}{{\cal M}}
\newcommand{\cp}{{\cal P}}
\newcommand{\cs}{{\cal S}}
\newcommand{\cz}{{\cal Z}}
\newcommand{\eps}{\varepsilon}
\newcommand{\ra}{\rightarrow}
\newcommand{\la}{\leftarrow}
\newcommand{\Ra}{\Rightarrow}
\newcommand{\dist}{\mbox{\rm dist}}
\newcommand{\bn}{{\mathbf N}}
\newcommand{\bz}{{\mathbf Z}}

\setlength{\parindent}{0pt}
%\setlength{\parskip}{2ex}
\newenvironment{proofof}[1]{\bigskip\noindent{\itshape #1. }}{\hfill$\Box$\medskip}

\renewcommand{\familydefault}{pnc}

\begin{document}

$\;$\hfill Suggested due date: 2018/03/15

\bigskip

\begin{center}
{\LARGE\bf Combinatorics: Homework 2}
\end{center}

\bigskip

\begin{hw}
Determine the size of the following set:
\[ \left\{m \in [2017] : \binom{2017}{m} \not\equiv 0 \pmod 3 \right\} \]
\end{hw}

\begin{sol}
	Let $n = (n_1n_2\dots n_k)_3$, $m = (m_1m_2\dots m_k)_3$.
	Then 
	$$
		\binom{n}{m} \equiv \binom{n_1}{m_1}\binom{n_2}{m_2}\cdots\binom{n_k}{m_k} \pmod{3}
	$$
	and $\binom{n}{m} \equiv 0 \pmod{3}$ if and only if exist $i$ such that $m_i > n_i$. So there are $(1+n_1)(1+n_2)\cdots(1+n_k)$ $m$ such that $\binom{n}{m} \not \equiv 0 \pmod{3} \quad (0 \leq m \leq n)$. So there are $(1+n_1)(1+n_2)\cdots(1+n_k)-1$ such $m$.
	
	$2017 = (2202201)_3$ and then there are $3^4 \times 2 -1= 161$ such $m$.
\end{sol}

\begin{hw}
Find the number of ordered pairs $(A, B, C)$ such that $A, B, C
\subseteq [n]$ and $A \cup B \cup C = [n]$. Justify your answer.
\end{hw}

\begin{sol}
	My answer is $7^n$.
	
	Let's consider every element $a$ that belongs to $[n]$. Because $A\cup B\cup C = [n]$, there are $7$ situations about whether $a \in A$, $a \in B$ and $a \in C$. So there are $7^n$ such $(A,B,C)$.
\end{sol}

\begin{hw}
Let $\mathcal{S}$ be the set of all the $k$-tuples of subsets of
$[n]$, i.e.
\[  \mathcal{S} = \{(A_1, A_2, ..., A_k): A_i \subseteq [n] \}.\]
Find a closed formula for
\[ \sum_{(A_1, A_2, ..., A_k) \in \mathcal{S}} |A_1 \cup A_2 \cup
  \dots \cup A_k|.\]
Justify your answer.
\end{hw}

\begin{sol}
	My answer is 
	$$
		\sum_{i = 0}^{n}i\binom{n}{i}(2^k-1)^i
	$$
	Let's first enumerate the size of set $A_1\cup A_2 \cup \cdots \cup A_k$, which is $i$ from $0$ to $n$. For the size $i$, there are $\binom{n}{i}$ sets. For each set, we can know that there are $(2^k-1)^i$ $k$-tuples to obtain such set(for similar reason of Problem 2). Finally, for each $A_1,A_2,\dots,A_k$, the size of their union is $i$. Sum them up, we can get the result.
\end{sol}

\begin{hw}
Give a combinatorial proof for the following identity
\[ \sum_{k=0}^{n} \binom{n}{k} \binom{n-k}{m-k} = 2^m \binom{n}{m} \]
\end{hw}

\begin{sol}
	Let's think about what the right side counts. It firstly choose a subset $A$ of $[n]$, and then choose a subset $B$ of $A$. And the number of $(A,B)$ is $2^m\binom{n}{m}$. 
	
	The left side counts such a number in a different way. Firstly, it enumerate the size of set $B$, named $k$ ($0 \leq k \leq n$). For each size $k$, there are $\binom{n}{k}$ such $B$. Then we need to know how many set $A$ that has the size $m$ and contains $B$. We can choose $m - k$ elements from the set $[n] \ B$ to get the rest elements of $A$. And the number if $\binom{n-k}{m-k}$. Finally, sum them up and we can get the number of all pairs $(A,B)$. 
\end{sol}

\begin{hw}
Give a combinatorial proof for the following equation. For any positive integers
$a$ and $b$,
\[ \sum_{i=0}^a \binom{a}{i} \binom{b+i}{a} = \sum_{i=0}^a \binom{a}{i}\binom{b}{i} 2^i .\]
\end{hw}

\begin{sol}
	Assume there exists two sets, called $A$ and $B$, and the size of $A$ is $a$ as well as the size of $B$ is $b$. 
	
	Let's see what the left side computes. Firstly, it chooses a subset of $A$ requiring its size is $i$ and we calls it $P$. Secondly, it chooses a subset of $P\cup B$ requiring its size is $a$ and we calls it $Q$. And the result of left side is the number of pairs $(P,Q)$.
	
	Let's see what the right side computes. Firstly, it enumerates $i$ as the size of $B\cap Q$ and chooses all the qualified $B-Q$, using $\binom{b}{i}$. Secondly, it chooses the elements that do not belong to $Q$ in $A$, using $\binom{a}{i}$ and the union of the rest part of $A$ and $B - Q$ is $Q$. So far, it has finished the enumerating of $Q$. We need to enumerate the number of $P$ when the $Q$ has determined. There are two requirement:
	\begin{itemize}
		\item $A \cap Q \subset P$.
		\item $P \subset A$.
	\end{itemize}
	We can enumerate every element of $A-(A\cap Q)$ to choose whether we put the element to $P$. The size of $A-(A\cap Q)$ is $i$ and there are $2^i$ such $P$. Then we have finished the enumeration of $(P,Q)$.
\end{sol}

\begin{hw} Can you colour all the non-negative integers into two
  colours, red and blue, such that each positive integer can be
  expressed as the sum of two different red
numbers in the same number of ways as the sum of two different blue numbers.
\end{hw}

\begin{sol}
	I have not solved this problem. But we find that if there exists such a color scheme, then it exists only one such scheme. We can construct such a scheme by considering whether $n$ has the same color with $0$. If $n$ can be expressed in the same number of two colors, we put $n$ apart from $0$. Otherwise, we put $n$ together with $0$. But I do not know if this process can last forever. If the difference of the two numbers greater than one, we will fail. I have write a program to have a try and this process can at least last to $10000$. So I think we "may" have such a coloring scheme.
\end{sol}

\begin{hw}
Can you colour the positive integers into some $k>1$ colours, such
that the numbers with any particular colour form an arithmetic
progression,
and the common difference of each colour is distinct?
\end{hw}

\begin{sol}
	It's not possible! Proof by contradiction:
	
	Assume that there exists a color scheme with $k$ colors. Define $C_i$ as the set of numbers whose color is $i$. We can find the color $p$ that has the minimum common difference and we call the set corresponds to the color is $C_p$ and call the common difference is $d_p$. Let $a$ be the minimum number in $C_p$. Let's consider the numbers $D = \{ a+1 + kd_p \mid k \in Z^+ \} $.  For all $i \in {1, 2, \dots, k}$, one of following two situations occurs:
	\begin{itemize}
		\item $D \cap C_i = \emptyset$.
		\item $D \cap C_i$ is a proper subset of $D$. And the elements in $D \cap C_i$ also form an arithmetic progression(The common difference is $lcm(d_p,d_i)$ and $lcm(d_p,d_i) > d_p$. 
	\end{itemize}
	We construct a mapping $f$ and let $f(a+1+kd_p) = k$. Then we can color all the positive numbers using at most $k-1$ colors(if $f(a+1+kd_p)$ has color $c$, then we give $k$ the color $c$). Because $C_p \cap D = \emptyset$, we have no number with color $p$. Therefore, the number of color decreases.
	
	Because $k$ is finite, we can repeat such a process and finally get that we can color all the positive numbers with only one color while the common difference is greater that one. But it's obvious not possible. 
	
	So there is not such a color scheme.
\end{sol}

\begin{hw} Does there exist a set of positive integers A, and two
  positive integers $N$ and $r$, such that for all integers $n>N$,
$n$ can be expressed in the same number of ways as the sum of two
numbers in $A$, i.e.
\[ |\{(a, b) : a, b \in A, a \leq b, a+b=n \}| = r ?\]
\end{hw}

\begin{sol}
	I have not solved this problem.
\end{sol}

\hfill by dyy

\end{document}
