\documentclass[12pt]{article}
\usepackage{amsmath}
\usepackage{amssymb}
\usepackage{amsthm}
\providecommand{\abs}[1]{\lvert#1\rvert}
\providecommand{\norm}[1]{\lVert#1\rVert}

\newtheorem{thm}{Theorem}
\newtheorem{lemma}[thm]{Lemma}
\newtheorem{fact}[thm]{Fact}
\newtheorem{cor}[thm]{Corollary}
\newtheorem{eg}{Example}
\newtheorem{ex}{Exercise}
\newtheorem{defi}{Definition}
\theoremstyle{definition}
\newtheorem{hw}{Problem}
\newenvironment{sol}
  {\par\vspace{3mm}\noindent{\it Solution}.}
  {\qed}

\newcommand{\ov}{\overline}
\newcommand{\cb}{{\cal B}}
\newcommand{\cc}{{\cal C}}
\newcommand{\cd}{{\cal D}}
\newcommand{\ce}{{\cal E}}
\newcommand{\cf}{{\cal F}}
\newcommand{\ch}{{\cal H}}
\newcommand{\cl}{{\cal L}}
\newcommand{\cm}{{\cal M}}
\newcommand{\cp}{{\cal P}}
\newcommand{\cs}{{\cal S}}
\newcommand{\cz}{{\cal Z}}
\newcommand{\eps}{\varepsilon}
\newcommand{\ra}{\rightarrow}
\newcommand{\la}{\leftarrow}
\newcommand{\Ra}{\Rightarrow}
\newcommand{\dist}{\mbox{\rm dist}}
\newcommand{\bn}{{\mathbf N}}
\newcommand{\bz}{{\mathbf Z}}

\setlength{\parindent}{0pt}
%\setlength{\parskip}{2ex}
\newenvironment{proofof}[1]{\bigskip\noindent{\itshape #1. }}{\hfill$\Box$\medskip}

\renewcommand{\familydefault}{pnc}

\begin{document}

$\;$\hfill Suggested due date: 2018/03/21

\bigskip

\begin{center}
{\LARGE\bf Combinatorics: Homework 3}
\end{center}

\bigskip

\section{Combinatorial identities}

\begin{hw}
Prove that for any $n \in \bn$,
\[\sum_{k \geq 0} \binom{n}{2k} \binom{2k}{k} 2^{n-2k} =
  \binom{2n}{n}.\]
There exists a combinatorial proof for this, but I don't suggest you try to find it.
\end{hw}

\begin{sol}
	Firstly, we can observe the left to get that:
	$$
		\binom{n}{2k}\binom{2k}{k} = \frac{n!}{(n-2k)!(2k)!}\frac{(2k)!}{k!k!} = \frac{n!}{k!(n-k)!}\frac{(n-k)!}{k!(n-2k)!} = \binom{n}{k}\binom{n-k}{n-2k}
	$$
	Then we only need to prove the following formula:
	$$
	\sum_{k \geq 0} \binom{n}{k}\binom{n-k}{n-2k} 2^{n-2k} = \binom{2n}{n}
	$$
	There exists a set $A = [2n]$ and we want to count the number of subset $S$ of $A$, whose size is $n$. It's obvious that the number is $\binom{2n}{n}$.
	
	Let's consider the left side. Firstly, it enumerate the set $P =  S\cap (S + {-n})$. Then we have determined $2 \mind 
\end{sol}

\begin{hw}
Prove that for any $m, n \in \bn$,
\[
\sum_{r \geq 0} \binom{2n}{2r-1} \binom{r-1}{m-1} = \binom{2n-m}{m-1} 2^{2n-2m+1}.
\]
Try to find a combinatorial proof for this one.
\end{hw}

\section{Practice on PIE}

In each problem, clearly specify what is the universe, what are the bad sets,
how to calculate the size of the bad sets, etc.

\begin{hw}
Complete the combinatorial proof of the following: For any positive integer
$n$,
\[
\sum_{k=0}^n(-1)^k \binom{2n-k}{k} 2^{2n-2k} = 2n+1.
\]
\end{hw}

\begin{hw}
Count the number of permutations $\pi$ of $[2n]$ such that $\pi(i)+ \pi(i+1) \neq 2n+1$ for all $1 \leq i \leq 2n-1$.
\end{hw}

\section{More for the pie day}

\begin{hw}
The Euler function $\phi(n)$ is defined to be the number of elements
in $[n]$ that are relatively prime to $n$. Define $f(n)= \sum_{i=1}^n
\phi(i)$.

Approximately how big is $f(n)$?
\end{hw}

\section{Some optional hard problems}

\begin{hw}
A {\em tournament} is a complete graph with exactly one direction on
each edge.
A {\em Hamilton path} in a graph is a path in the graph that visits
every vertex exactly once.

Prove that, in any tournament,

(a) there exists at least one Hamilton path;

(b) the number of Hamilton paths is odd.

\end{hw}

\begin{hw}
Suppose $G$ is a connected graph. A {\em spanning subgraph} of $G$ is
a subgraph that connects all the vertices of $G$. Prove that,
$G$ has an odd number of spanning subgraphs if and only if
$G$ is bipartite.
\end{hw}

\end{document}
