\documentclass[12pt]{article}
\usepackage{amsmath}
\usepackage{amssymb}
\usepackage{amsthm}
\providecommand{\abs}[1]{\lvert#1\rvert}
\providecommand{\norm}[1]{\lVert#1\rVert}

\newtheorem{thm}{Theorem}
\newtheorem{lemma}[thm]{Lemma}
\newtheorem{fact}[thm]{Fact}
\newtheorem{cor}[thm]{Corollary}
\newtheorem{eg}{Example}
\newtheorem{ex}{Exercise}
\newtheorem{defi}{Definition}
\theoremstyle{definition}
\newtheorem{hw}{Problem}
\newenvironment{sol}
  {\par\vspace{3mm}\noindent{\it Solution}.}
  {\qed}

\newcommand{\ov}{\overline}
\newcommand{\cb}{{\cal B}}
\newcommand{\cc}{{\cal C}}
\newcommand{\cd}{{\cal D}}
\newcommand{\ce}{{\cal E}}
\newcommand{\cf}{{\cal F}}
\newcommand{\ch}{{\cal H}}
\newcommand{\cl}{{\cal L}}
\newcommand{\cm}{{\cal M}}
\newcommand{\cp}{{\cal P}}
\newcommand{\cs}{{\cal S}}
\newcommand{\cz}{{\cal Z}}
\newcommand{\eps}{\varepsilon}
\newcommand{\ra}{\rightarrow}
\newcommand{\la}{\leftarrow}
\newcommand{\Ra}{\Rightarrow}
\newcommand{\dist}{\mbox{\rm dist}}
\newcommand{\bn}{{\mathbf N}}
\newcommand{\bz}{{\mathbf Z}}

\setlength{\parindent}{0pt}
%\setlength{\parskip}{2ex}
\newenvironment{proofof}[1]{\bigskip\noindent{\itshape #1. }}{\hfill$\Box$\medskip}

\renewcommand{\familydefault}{pnc}

\begin{document}

$\;$\hfill Suggested due date: 2018/03/21

\bigskip

\begin{center}
{\LARGE\bf Combinatorics: Homework 3}
\end{center}

\bigskip

\section{Combinatorial identities}

\begin{hw}
Prove that for any $n \in \bn$,
\[\sum_{k \geq 0} \binom{n}{2k} \binom{2k}{k} 2^{n-2k} =
  \binom{2n}{n}.\]
There exists a combinatorial proof for this, but I don't suggest you try to find it.
\end{hw}

\begin{sol}
	Firstly, we can observe the left to get that:
	$$
		\binom{n}{2k}\binom{2k}{k} = \frac{n!}{(n-2k)!(2k)!}\frac{(2k)!}{k!k!} = \frac{n!}{k!(n-k)!}\frac{(n-k)!}{k!(n-2k)!} = \binom{n}{k}\binom{n-k}{n-2k}
	$$
	Then we only need to prove the following formula:
	$$
	\sum_{k \geq 0} \binom{n}{k}\binom{n-k}{n-2k} 2^{n-2k} = \binom{2n}{n}
	$$
	
	Assuming there exists a 01 string whose length is $2n$. We call the first half $s_1$ and the second half $s_2$. The right side of the formula count the number of such string that contains $n$ $1$. 
	
	Let's consider what the left side of the formula count. Firstly, it enumerate the intersection of $s_1$ and $s_2$ that contains $k$ $1$. Then there are totally $2k$ $1$ in the whole string and we need to choose another $n - 2k$ $1$ to be $1$. We can select $n-2k$ places in the rest $n-k$ places. For any chosen place, it can be in either $s_1$ or $s_2$. There are totally $2^{n-2k}$ chooses. Finally, we get the left side of the formula.
\end{sol}

\begin{hw}
Prove that for any $m, n \in \bn$,
\[
\sum_{r \geq 0} \binom{2n}{2r-1} \binom{r-1}{m-1} = \binom{2n-m}{m-1} 2^{2n-2m+1}.
\]
Try to find a combinatorial proof for this one.
\end{hw}

\section{Practice on PIE}

In each problem, clearly specify what is the universe, what are the bad sets,
how to calculate the size of the bad sets, etc.

\begin{hw}
Complete the combinatorial proof of the following: For any positive integer
$n$,
\[
\sum_{k=0}^n(-1)^k \binom{2n-k}{k} 2^{2n-2k} = 2n+1.
\]
\end{hw}

\begin{sol}
	Let $\{a_n\}$ be a sequence of $R$ and $B$ and its length is $2n$. A sequence is good if and only if it does not contain $RB$. Let $B_i$ ($1 \leq i \leq 2n-1$) be the set that contains all the bad sequences that $a_i = R$ and $a_{i+1} = B$.
	
	By enumerating the position of the first $B$, we can know that there are totally $2n+1$ good sequences. 
	
	By PIE, we know that the number of all the good sequences is:
	\[ \begin{split}
		&   \sum_{S \subset [2n-1]}(-1)^{\abs{S}}B_{S}  
		    = \sum_{k = 0}^{2n-1}\sum_{\substack{S\subset [2n-1]\\ \abs{S}=k}} (-1)^{\abs{S}}B_{S} \\
		&   = \sum_{k = 0}^{2n-1}(-1)^k\sum_{\substack{S\subset [2n-1]\\ \abs{S}=k}}B_{S} 
		    =\sum_{k = 0}^{2n-1}(-1)^k \binom{2n-k}{k}2^{2n-2k}
	\end{split} \]
	
	The last equation may need some explanation. If the exist $i$ such that $i \in S$ and $i+1 \in S$, $B_S$ is empty. Because we can not assign two different values to the same place. If there does not exist such $i$, we can use $\binom{2n-2k + k}{k}$ to find the number of such $S$. For each $S$, we have determined the values of $2k$ places and we can assign $R$ or $B$ to other $2n-2k$ places and there are totally $2^{2n-2k}$ ways. 
	
	
\end{sol}

\begin{hw}
Count the number of permutations $\pi$ of $[2n]$ such that $\pi(i)+ \pi(i+1) \neq 2n+1$ for all $1 \leq i \leq 2n-1$.
\end{hw}

\begin{sol}
	Let $B_i (1 \leq i \leq 2n-1)$ be the set of permutation $\pi$ satisfying $\pi(i) + \pi(i+1) = 2n + 1$. Then the answer is 
	$$
		\sum_{S \subset [2n-1]}(-1)^{\abs{S}}B_S = \sum_{k = 0}^{2n-1}(-1)^k\binom{2n-k}{k}\frac{n!}{(n-k)!}(n-2k)!
	$$
	For a $B_S$, if $i,i+1 \in S$, the $B_S$ is an empty set. So there are $\binom{2n-k}{k}$ nonempty $B_S$. For each $S$ that has the size $k$, we can enumerate the minor number in each of the $k$ couple places. Then we have $\frac{n!}{(n-k)!}$ schemes. For the rest $n-2k$ numbers, we can random places them and there are $(n-2k)!$ schemes. 
\end{sol}

\section{More for the pie day}

\begin{hw}
The Euler function $\phi(n)$ is defined to be the number of elements
in $[n]$ that are relatively prime to $n$. Define $f(n)= \sum_{i=1}^n
\phi(i)$.

Approximately how big is $f(n)$?
\end{hw}

\section{Some optional hard problems}

\begin{hw}
A {\em tournament} is a complete graph with exactly one direction on
each edge.
A {\em Hamilton path} in a graph is a path in the graph that visits
every vertex exactly once.

Prove that, in any tournament,

(a) there exists at least one Hamilton path;

(b) the number of Hamilton paths is odd.

\end{hw}

\begin{hw}
Suppose $G$ is a connected graph. A {\em spanning subgraph} of $G$ is
a subgraph that connects all the vertices of $G$. Prove that,
$G$ has an odd number of spanning subgraphs if and only if
$G$ is bipartite.
\end{hw}

\end{document}
