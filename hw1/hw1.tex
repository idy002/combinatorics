
\documentclass[12pt]{article}
\usepackage{amsmath}
\usepackage{amssymb}
\usepackage{amsthm}
\providecommand{\abs}[1]{\lvert#1\rvert}
\providecommand{\norm}[1]{\lVert#1\rVert}

\newtheorem{thm}{Theorem}
\newtheorem{lemma}[thm]{Lemma}
\newtheorem{fact}[thm]{Fact}
\newtheorem{cor}[thm]{Corollary}
\newtheorem{eg}{Example}
\newtheorem{ex}{Exercise}
\newtheorem{defi}{Definition}
\theoremstyle{definition}
\newtheorem{hw}{Problem}
\newenvironment{sol}
{\par\vspace{3mm}\noindent{\it Solution}.}
{\qed}

\newcommand{\ov}{\overline}
\newcommand{\cb}{{\cal B}}
\newcommand{\cc}{{\cal C}}
\newcommand{\cd}{{\cal D}}
\newcommand{\ce}{{\cal E}}
\newcommand{\cf}{{\cal F}}
\newcommand{\ch}{{\cal H}}
\newcommand{\cl}{{\cal L}}
\newcommand{\cm}{{\cal M}}
\newcommand{\cp}{{\cal P}}
\newcommand{\cs}{{\cal S}}
\newcommand{\cz}{{\cal Z}}
\newcommand{\eps}{\varepsilon}
\newcommand{\ra}{\rightarrow}
\newcommand{\la}{\leftarrow}
\newcommand{\Ra}{\Rightarrow}
\newcommand{\dist}{\mbox{\rm dist}}
\newcommand{\bn}{{\mathbf N}}
\newcommand{\bz}{{\mathbf Z}}

\setlength{\parindent}{0pt}
%\setlength{\parskip}{2ex}
\newenvironment{proofof}[1]{\bigskip\noindent{\itshape #1. }}{\hfill$\Box$\medskip}

\renewcommand{\familydefault}{pnc}

\begin{document}
	
	$\;$\hfill Suggested due date: 2018/03/07
	
	\bigskip
	
	\begin{center}
		{\LARGE\bf Combinatorics: Homework 1}
	\end{center}
	
	\bigskip
	
	\begin{hw}
		
		How many odd numbers are there in the 2018-th row of the Pascal triangle?
		
		NOTE: The 2018-th row has the binomial numbers $\dbinom{2018}{r}$.
	\end{hw}
	
	\begin{hw}
		The sets are defined as
		\[
		A_0 = \{0\}, \;\; A_1 = \{1\}, \;\; A_{i+2}=(A_1 + A_{i+1})\Delta A_{i} \; (i \geq 0),
		\]
		where for two sets of numbers $A$ and $B$, $A+B = \{a+b : a \in A, b \in B\}$.
		
		(a) Enumerate $A_i$ for $i=0, 1, \dots, 15$.
		
		(b) Prove that $|A_n| = 1$ for infinitely many $n$.
	\end{hw}
	
	\begin{hw}
		$n$ points form a circle, on one point there are $n$ chips. In each step, one can pick a point where there are at least two chips, and send one to each of its neighbours.
		The game stops when no point has more than 1 chip.
		
		(a) Prove that when $n$ is even, the game never stops.
		
		(b) Prove that when $n$ is odd, the game always stops.
	\end{hw}
	
	\begin{hw}
		With cost $\alpha$ ($\alpha$ can be any positive real number), you can buy a strip of land between any two parallel lines (including the lines) in the plane of your choice, as long as the distance between the two lines is at most $\alpha$. It is easy to see that you can cover all the points in the unit circle with total cost 2. Prove that you cannot do it with less cost.
		
	\end{hw}
	
\end{document}
