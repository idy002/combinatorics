\documentclass[12pt]{article}
%\usepackage{xeCJK}
\usepackage{amsmath}
\usepackage{amssymb}
\usepackage{amsthm}
\providecommand{\abs}[1]{\lvert#1\rvert}
\providecommand{\norm}[1]{\lVert#1\rVert}

\newtheorem{thm}{Theorem}
\newtheorem{lemma}[thm]{Lemma}
\newtheorem{fact}[thm]{Fact}
\newtheorem{cor}[thm]{Corollary}
\newtheorem{eg}{Example}
\newtheorem{ex}{Exercise}
\newtheorem{defi}{Definition}
\theoremstyle{definition}
\newtheorem{hw}{Problem}
\newenvironment{sol}
{\par\vspace{3mm}\noindent{\it Solution}.}
{\qed}

\newcommand{\ov}{\overline}
\newcommand{\cb}{{\cal B}}
\newcommand{\cc}{{\cal C}}
\newcommand{\cd}{{\cal D}}
\newcommand{\ce}{{\cal E}}
\newcommand{\cf}{{\cal F}}
\newcommand{\ch}{{\cal H}}
\newcommand{\cl}{{\cal L}}
\newcommand{\cm}{{\cal M}}
\newcommand{\cp}{{\cal P}}
\newcommand{\cs}{{\cal S}}
\newcommand{\cz}{{\cal Z}}
\newcommand{\eps}{\varepsilon}
\newcommand{\ra}{\rightarrow}
\newcommand{\la}{\leftarrow}
\newcommand{\Ra}{\Rightarrow}
\newcommand{\dist}{\mbox{\rm dist}}
\newcommand{\bn}{{\mathbf N}}
\newcommand{\bz}{{\mathbf Z}}

\setlength{\parindent}{0pt}
%\setlength{\parskip}{2ex}
\newenvironment{proofof}[1]{\bigskip\noindent{\itshape #1. }}{\hfill$\Box$\medskip}

\renewcommand{\familydefault}{pnc}

\begin{document}
	
	$\;$\hfill Suggested due date: 2018/03/07
	
	\bigskip
	
	\begin{center}
		{\LARGE\bf Combinatorics: Homework 1}
	\end{center}
	
	\bigskip
	
	\begin{hw}
		According to Lucas's theorem, we have
		$$
		\binom{n}{i} 
		\equiv \binom{(n_1n_2\dots n_k)_2}{(i_1i_2\dots i_k)_2}
		\equiv \binom{n_1}{i_1}\binom{n_2}{i_2}\cdots\binom{n_k}{i_k} \pmod 2
		$$
		where $(n_1n_2\dots n_k)_2$ and $(i_1i_2\dots i_k)_2$ are the binary expressions of $n$ and $i$.
		
		$\binom{n}{i}$ is odd if and only if $i_k \leq n_k$ for all $k$, so the number of such $i$ is $2^{cnt}$ where $cnt$ is the number of $1$ in the binary expression of $n$.
		$2018 = (0111 \; 1110 \; 0010)_2$ so the answer is $2^7 = 128$.
	\end{hw}
	
	\begin{hw}
		\textbf{(a)} If we use a bit string to express a set, we have: \\
			$A_{0} = 0000000000000001$ \\
			$A_{1} = 0000000000000010$ \\
			$A_{2} = 0000000000000101$ \\
			$A_{3} = 0000000000001000$ \\
			$A_{4} = 0000000000010101$ \\
			$A_{5} = 0000000000100010$ \\
			$A_{6} = 0000000001010001$ \\
			$A_{7} = 0000000010000000$ \\
			$A_{8} = 0000000101010001$ \\
			$A_{9} = 0000001000100010$ \\
			$A_{10} = 0000010100010101$ \\
			$A_{11} = 0000100000001000$ \\
			$A_{12} = 0001010100000101$ \\
			$A_{13} = 0010001000000010$ \\
			$A_{14} = 0101000100000001$ \\
			$A_{15} = 1000000000000000$ \\

		\textbf{(b)}
		For simplification, let's define $B_0 = \emptyset$ and $B_{i+1} = \{a + 1 \vert a \in A_i\}$ for all $i \geq 0$, $S^{(n)} = S + \{1\} + \{1\} + \cdots + \{1\}$ (add $n$ times) and $AB = A\Delta B$ . So $B_1 = \{1\}$, $B_2 = \{2\}$, $B_3 = \{ 1, 3\}$ and so on. 
		
		Let's define the proposition $P(N)$ be :
		$$
			B_{2^N+i} = B_{2^N-i}B_i^{(2^N)} \text{ for all } 1 \leq i \leq 2^N - 1
		$$ and $$
			B_{2^N} = \{2^N\}
		$$
		We can check and get that $P(N)$ holds for $N = 0, 1$.
		
		Assuming $P(N)$ is true, let's prove $P(N+1)$ is also true:
		\begin{itemize}
			\item $B_{2^{N+1}} = \{2^{N+1}\}$: 
				\[ \begin{split}
				B_{2^{N+1}} & = B_{2^{N+1}-1}^{(1)}B_{2^{N+1}-2} \\
							& = (B_1B_{2^N-1}^{(2^N)})^{(1)}B_2B_{2^N-2}^{(2^N)} \\
							& = B_1^{(1)}B_2B_{2^N-1}^{(2^N+1)}B_{2^N-2}^{2^N} \\
							& = B_{2^N-1}^{(2^N+1)}B_{2^N-2}^{2^N} \\
							& = (B_{2^N})^{(2^N)} \\
							& = (\{2^N\})^{(2^N)} \\
							& = \{2^{N+1}\}
				\end{split} \]
			\item $i = 1$:
				\[ \begin{split}
				B_{2^{N+1}+1} & = B_{2^{N+1}}^{(1)}B_{2^{N+1}-1} \\
							  & = \{2^{N+1} + 1\}B_{2^{N+1}-1} \\
							  & = B_{1}^{(2^{N+1})}B_{2^{N+1}-1}
				\end{split} \]
			\item $2 \leq i \leq 2^N - 1$:(This part itself is also an induction proof)
				 \[ \begin{split}
					 B_{2^{N+1}+i} 
					 & = (B_{2^{N+1}-i+1}B_{i-1}^{(2^{N+1})})^{(1)}B_{2^{N+1}-i+2}B_{i-2}^{2^{N+1}}  \\
					 & = (B_{2^{N+1}-i+1})^{(1)}B_{2^{N+1}-i+2}(B_{i-1}B_{i-2})^{(2^{N+1})} \\
					 & = B_{2^{N+1}-i}B_{i}^{2^{N+1}}
				 \end{split} \]
		\end{itemize}
		So $\abs{A_{2^n-1}} = \abs{B_{2^n}} = 1$ for all $n >= 1$.
	\end{hw}
	
	\begin{hw}
		\textbf{(b)}
			Let's define a state function $g(s)$ equal to the sum of the distances between all the $n$ points for the state $s$. Explicitly, 
			$$
				g(s) = \sum_{i = 0}^{n}\sum_{j = i + 1}^{n}s[i]s[j]min(j - i, n - j + i)
			$$ 
			where $s[0], s[1], \dots, s[n-1]$ are the number of points at the places $0, 1, \cdots, n - 1$.
			
			It's easy to observe the fact that any movement will cause $g(s)$ increasing and $g$ will stop increasing when no movement can be done.(Only when $n$ is odd, $g$ has this property)
	\end{hw}
	
	\bigskip
	
	\bigskip
	
	\bigskip 
	
	I've known the answers of the rest problems from Wang Tianzhe. 
	
\end{document}
